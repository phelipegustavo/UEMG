\documentclass{beamer}
\usepackage[utf8]{inputenc} %colocar os acentos padrão brasileiro
\usepackage[T1]{fontenc}
\usepackage[portuguese]{babel}
\usepackage{graphicx}
\usepackage{hyphenat}
\hyphenation{mate-mática recu-perar}
%\documentclass[handout,t]{beamer}
\batchmode
% \usepackage{pgfpages}
% \pgfpagesuselayout{4 on 1}[letterpaper,landscape,border shrink=5mm]
\usepackage{amsmath,amssymb,enumerate,epsfig,bbm,calc,color,ifthen,capt-of}
\usetheme{Berlin}
%\usetheme{AnnArbor}
\usecolortheme{mit}


%Sumário do pdf **************************************

\title{Escolha do Sistema operacional}
\author[ALDO-GUSTAVO-PHELIPE]{Aldo \and Gustavo \and Phelipe}
\date{\today}

%*************************************************************
%imagens que ficarão fixas em baixo
\pgfdeclareimage[height=0.5cm]{uemg-logo}{uemg-logo.pdf} % tamanho da imagem
\logo{\pgfuseimage{uemg-logo}\hspace*{0.3cm}} %margem direita


\AtBeginSection[]
{
	
	
	\begin{frame}<beamer>
	\frametitle{Sumário}
	\tableofcontents[currentsection]
\end{frame}
}



\logo{\includegraphics[scale=0.2]{uemg-logo.jpg}}










\begin{document} %INICIO DOCUMENTO


\frame{\titlepage}
\section[Sumário]{}

\begin{frame}{Sumário}
\tableofcontents
\end{frame}


%------------------------------------
\section{Manjaro}
\subsection{Prós}
\begin{frame}{Prós}
\begin{itemize}
\item <1->Interface amigável.
\item <2->Comunidade ativa.
\item <3->Sistema operacional estável.
\item <4->Rolling Realese( faz atualizações para melhorias).

\end{itemize}
\end{frame}

\subsection{Contras}
\begin{frame}{Contras}
\begin{itemize}
\item <1->Suporte a computadores antigos(pode deixar computador lento).
\item <2->Suporte para alguns drive com placa de vídeo, webcam, placa de som podem não ser reconhecido quando computador inicia.




\end{itemize}
\end{frame}



%--------------------------------------
\section{Conclusão}
\subsection{Conclusão}
\begin{frame}{Conclusão}
Existem centenas de distribuições linux, basta  pesquisar para encontrar a mais adequada a seus objetivos, e que ao mesmo tempo se adéque ao ao hardware especifico que esta utilizando.

\end{frame}

%\section{Referencias}

%\begin{frame}{Referencias}
%\subsection{Referencias}
%   \begin{itemize}
%\item Browse \url{https://tecnologia.uol.com.br/especiais/ultnot/2005/08/18/ult2888u81.jhtm}
%\item Browse \url{http://meupinguim.com/ubuntu-vs-mint-qual-melhor}
%     \end{itemize}
% \end{frame}
\end{document}
